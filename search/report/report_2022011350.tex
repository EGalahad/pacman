\documentclass{article}
\usepackage[utf8]{inputenc}
\usepackage{listings}
\usepackage{graphicx}
\usepackage{float}
\usepackage{xcolor}
\usepackage{amsmath}
\usepackage{enumerate}

\title{\bf\Large  AI 2022 Fall\\Assignment 2 Search Submission}
%%%%%%%%%%%%%%%%%%%%%%%%%%%%%%%%%%%%%%
%% DON'T forget to change this part %%
\author{\bf Name: Weng Haoyang \qquad Student ID: 2022011350}
%%%%%%%%%%%%%%%%%%%%%%%%%%%%%%%%%%%%%%

\begin{document}
\maketitle
\section{Writing Component}

\subsection{Search with Heuristics}
\paragraph*{(a)}
\[\mathcal{O} (n^3)\]

\paragraph*{(b)}
\[ h_i(x_i, y_i) = |x_i - (n - i + 1)| + |y_i - n|\]
Of course,  Manhattan distance is admissible.

May not with other cars, because hopping over cars get a jump of \(2\) with cost of \(1\).

\paragraph*{(c)}
% item use roman number
\begin{enumerate}[(i)]
    \item No.
    \item Yes, because hopping only halves the cost needed to the goal.
    \item No.
    \item No.
    \item No.
\end{enumerate}

\subsection{Adjusted Heuristics}
\begin{figure}[H]
    \centering
    \includegraphics[width=\textwidth]{../image/counter_example.png}
    \caption{counetr example}
    \label{fig:counter}
\end{figure}

\section{Programming Component: Search in Pacman}
\setcounter{subsection}{2}
\subsection{Question 1: Depth First Search}
\begin{figure}[H]
    \centering
    \includegraphics[width=\textwidth]{../image/q1.png}
    \caption{q1}
    \label{fig:q1}
\end{figure}

\subsection{Question 2: Breadth First Search}
\begin{figure}[H]
    \centering
    \includegraphics[width=\textwidth]{../image/q2.png}
    \caption{q2}
    \label{fig:q2}
\end{figure}

\subsection{Question 3: Uniform Cost Search}
\begin{figure}[H]
    \centering
    \includegraphics[width=\textwidth]{../image/q3.png}
    \caption{q3}
    \label{fig:q3}
\end{figure}

\subsection{Question 4: A\* Search}
In this problem I implemented A\* search with null heuristic.

\begin{figure}[H]
    \centering
    \includegraphics[width=\textwidth]{../image/q4.png}
    \caption{q4}
    \label{fig:q4}
\end{figure}

\subsection{Question 5: Finding All the Corners}
In this problem I defined state for the problem by supplying the state with a 4-tuple of bool to record which corner have been reached

\begin{figure}[H]
    \centering
    \includegraphics[width=\textwidth]{../image/q5.png}
    \caption{q5}
    \label{fig:q5}
\end{figure}

\subsection{Question 6: Corners Problem: Heuristic}
In this problem I designed a heuristic for 4 corner-search problem.
The heuristic is the minimum cost to reach the goal in a relaxed problem, i.e. without walls, 
 and I consider the four cases separately.

\begin{figure}[H]
    \centering
    \includegraphics[width=\textwidth]{../image/q6.png}
    \caption{q6}
    \label{fig:q6}
\end{figure}

\subsection{Question 7: Eating All The Dots}
I designed four heuristics for this problem, and I will prove that they are consistent.

\paragraph*{(a)}
The first heuristic is the Manhattan distance to the closest dot.

We need to prove \(h(A) \leq h(C) + \text{cost} (A to C)\). 
If \(A\) is a dot, then \(h(A) = 0\). 
If \(C\) is the closest dot of \(A\), then \(h(C) = 0\), \(h(A) = \text{cost}(A to C) = 1\).
Leaving us with the case where \(C\) is not the closest dot of \(A\).
Taking one step from \(A\) to \(C\) can only cause a change of magnitude \(1\) in the Manhattan distance to every dot,
so the heuristic is consistent.
\paragraph*{(b)}
The second heuristic is the Manhattan distance to the furthest dot.

Proof is same as above.
\paragraph*{(c)}
The third heuristic is the distance between two closest dots and the smaller one of the Manhattan distances to the two.

If \(C\) is not a dot, then the two closest dots are not different in evaluating the heuristic, so the proof is same.

If \(C\) is a dot, and \(C\) is one of the two closest dots, then \(h(A) = \text{distance of closest two dots in some set} + 1\) and \(h(C) = \text{distance of closest two dots in some set except C} + \text{some distance larger than 1}\). 
So the heuristic is consistent.
\paragraph*{(d)}
The fourth heuristic is the distance between two furthest dots and the smaller one of the Manhattan distances to the two.

The proof is similar.

\begin{figure}[H]
    \centering
    \includegraphics[width=\textwidth]{../image/q7_1.png}
    \caption{q7 (the heuristic used to test is 4)}
    \label{fig:q7}
\end{figure}

\subsection{Question 8: Suboptimal Search}

\begin{figure}[H]
    \centering
    \includegraphics[width=\textwidth]{../image/q8.png}
    \caption{q8}
    \label{fig:q8}
\end{figure}

The counterexample of closest dot search is inspired by the corners problem. The setting is three dots on the corner and starting point close to the dot in the middle.

\begin{figure}[H]
    \centering
    \includegraphics*[width=\textwidth]{../image/suboptimal.png}
    \caption{suboptimal}
    \label{fig:suboptimal}
\end{figure}

\end{document}